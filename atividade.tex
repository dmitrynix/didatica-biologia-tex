\documentclass[11pt]{article}

\usepackage[utf8]{inputenc}
\usepackage[portuges]{babel}
\usepackage[T1]{fontenc}
\usepackage{enumerate}

\begin{document}
  \begin{center}
    Atividade
  \end{center}

  \begin{enumerate}
    \item Que tipo de indivíduo é originado pela união de gametas portadores do mesmo tipo de alelo de um gene ?
      \begin{enumerate}[(a)]
        \item Heterozigoto;
        \item Homozigoto;
        \item Dominante;
        \item Recessivo.
      \end{enumerate}


    \item No cruzamento de dois indivíduos heterozigóticos  Aa, espera-se obter:
      \begin{enumerate}[(a)]
        \item Apenas indivíduos Aa;
        \item Indivíduos AA e AA, na proporção de 3:1, respectivamente;
        \item Indivíduos AA e aa, na proporção 1:1 respectivamente;
        \item Indivíduos AA, Aa e aa, na proporção de 1:2:1, respectivamente.
      \end{enumerate}

    \item Duas pessoas do grupo sanguíneo AB podem ter apenas filhos de sangue tipo:
      \begin{enumerate}[(a)]
        \item AB;
        \item A e B;
        \item O;
        \item A, B e AB.
      \end{enumerate}

    \item Uma espécie com sistema de determinação do sexo do tipo XY produz que tipos de gametas ?
      \begin{enumerate}[(a)]
        \item Dois tipos de óvulo e um tipo de espermatozoide;
        \item Dois tipos de óvulo e dois tipos de espermatozoide;
        \item Um tipo de óvulo e um tipo de espermatozoide;
        \item Um tipo de óvulo e dois tipos de espermatozoide.
      \end{enumerate}
  \end{enumerate}
\end{document}
